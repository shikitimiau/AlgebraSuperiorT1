\section{}
Demuestra que para cualquiera conjunto $A$ y $B$, se tiene que:
\begin{enumerate}[(a)]
    \item $A \subseteq B$ si y solo si $ A \setminus B = \varnothing$; \newline
    $(\implies) A \subseteq B \implies A \setminus B = \varnothing$\newline
    Demostración por contradicción:\newline
    $A \subseteq B(1) \wedge A \setminus B \neq \varnothing(2)$\newline
	Por definición podemos expresar a  $(1)$ y $(2)$ como: \newline
	$A \subseteq B \coloneqq \forall x (x \in A \implies x \in B)$  y $x \in A \setminus B \coloneqq (x \in A \wedge x \notin B) !$\newline
	Esto nos lleva a una contradicción ya que la diferencia de $A$ y $B$, $x \in A$ y $x \notin B$, pero se dijo que  $\forall x(x \in A \implies x \in B)$ por lo que no puede ser que $x \in A$ y $x \notin B$.\\\\
    
    $(\Longleftarrow) A \setminus B = \varnothing \implies A \subseteq B$\newline
    Demostración por contrapositiva:\newline
	Si $A \subsetneqq B \implies A \setminus B \neq \varnothing$, por definición de diferencia tenemos que: \newline
	$x \in A - B \coloneqq (x \in A \wedge x \notin B)$ , por hipótesis tenemos que: \newline
	$A \subsetneqq B \coloneqq \exists x (x \in A \wedge x \notin B )$ por lo que si a esto le aplicamos una diferencia nos quedaría que: \newline
	 $x \in A \wedge x \notin B$ (def) y  como $\exists x (x \in A \wedge x \notin B )$ entonces $A \setminus B \neq 0$ lo que demuestra que:\newline
	 $A \setminus B = \varnothing \implies  A \subseteq B$    
    
    
    
     \item $B \subseteq A$ si y solo si $ (A \setminus B) \cup B = A$; \newline
    $(\implies) B \subseteq A \implies (A \setminus B) \cup B = A$\newline
    Demostración por contradicción:\newline
    Si $B \subseteq A(1) \wedge (A \setminus B) \cup B \neq A(2)$, por definición, sabemos que:\newline
    $B \subseteq A \coloneqq \forall x (x \in B \implies x \in A)$ \newline
    y que $(A \setminus B) \cup B \neq A$ es $x \in (\{x \in A \wedge x \notin B \} \lor x \in B)$ pero $x \notin A !$.\newline
    Esto es una contradicción ya que si $\forall x(x \in B \implies x \in A)$ \newline
    y $(x \in A \wedge x \in B)$, \newline
    no puede ser $x \notin A$ ya que en ambos casos $x \in A$. \newline
   $\therefore  B \subseteq A \implies (A \setminus B) \cup B = A$.\\\\
   
   $(\Longleftarrow) (A \setminus B) \cup B = A \implies B \subseteq A$\newline
   Por definición podemos expresar a  $(A \setminus B) \cup B = A$ como:\newline
    $x \in ( \{ x \in A \wedge x \notin B\} \lor x \in B) = A$ \newline
    por lo tanto $( \{x \in A \wedge x \notin B\} \lor x \in B) \subseteq A(1)$ \newline
    y $A \subseteq( \{ x \in A \wedge x \notin B \} \lor x \in B)$,\newline
     por $(1)$ tenemos dos casos $\{ x \in A \wedge x \notin B\} \subseteq A$ que es cierto ya que por teorema de reflexibilidad(visto en clase ) $A \subseteq A$ \newline
     y tenemos el segundo caso $B \subseteq A$, lo que completa la demostración.


  
  \end{enumerate}