\section{}
\noindent 10. Sean $\mathbb{N},$ $A = \{n \in \mathbb{N} : n^2 > 2n-1\}$ y $B = \{n \in \mathbb{N} : n^2 = 2n+3\}$. Determine los siguientes conjuntos: $A \cap B, A \cup B, A^c , B^c , (A\cap B)^c $ y $A^c \cup B^c$\\

\begin{center}
\begin{tabular}{r l}
$A$ & $= \{n \in \mathbb{N} : n^2 - 2n-1 > 0\}$\\
$A$ & $= \{n \in \mathbb{N} : (n-1)^2 > 0\}$
\end{tabular}

\end{center}
El cuadrado de un número siempre es igual o mayor a $0$, por lo que el enunciado es falso cuando $(n-1)^2 = 0$\\
\begin{center}
$\mathbf{\therefore \; A = \{n \in \mathbb{N} : n \neq 1\}}$
\end{center}
Para B:
\begin{center}
\begin{tabular}{r l}
$B$ & $= \{n \in \mathbb{N} : n^2 -2n -3= 0\}$\\
$B$ & $= \{n \in \mathbb{N} : (n-3)(n+1) = 0\}$\\
$\mathbf{\therefore \; B}$ & $= \mathbf{\{n \in \mathbb{N} : n=-1\; \vee \; n = 3\}}$
\end{tabular}

\end{center}

\paragraph{$\mathbf{A \cap B}$}

	\subparagraph{$= \{n \in \mathbb{N} : n \in A, n \in B\}$}

	\subparagraph{$= \{n \in \mathbb{N} : n \neq -1 \; \wedge \; (n=-1 \; \vee \; n = 3)\}$}
	\subparagraph{$= \{n \in \mathbb{N} : n=-1 \; \vee \; n = 3\}$}	
\paragraph{$\mathbf{A \cup B}$}
	
	\subparagraph{$= \{n \in \mathbb{N} : n \in A \; \vee \; n \in B\}$}
	\subparagraph{$= \{n \in \mathbb{N} : n \neq 1 \; \vee \; n = -1 \vee n=3\}$}
	
\paragraph{$\mathbf{A^c}$}
	
	\subparagraph{$= \{n \in \mathbb{N} : n \notin A\}$}
	\subparagraph{$= \{n \in \mathbb{N} : n = 1\}$}
\newpage		
\paragraph{$\mathbf{B^c}$}
	\subparagraph{$= \{n \in \mathbb{N} : n \notin B\}$}
	\subparagraph{$= \{n \in \mathbb{N} :n \neq -1 \; \wedge \; n \neq 3\}$}
\paragraph{$\mathbf{(A\cap B)^c}$}
	
	\subparagraph{$= \{n \in \mathbb{N} : n \notin (A \cup B)\}$}
	\subparagraph{$= \{n \in \mathbb{N} : n \neq -1,\; n \neq 3\}$}
	
\paragraph{$\mathbf{A^c \cap B^c}$}
	\subparagraph{$= \{n \in \mathbb{N} : n \notin A \; \vee \; n \notin B\}$}
	\subparagraph{$= \{n \in \mathbb{N} :n = 1 \; \vee \; (n \neq -1  \; \wedge \; n \neq 3)\}$}
	\subparagraph{$= \{n \in \mathbb{N} : n \neq -1 \; \wedge \; n \neq 3\}$}