\section{}
\noindent 20. Sean $C = \{4k+1 : k \in \mathbb{N}\}$ y $D = \{2k+1 : k \in \mathbb{N}\}$. Pruebe que $C \subseteq D$ y determine el conjunto $D/C$.

\paragraph{PD. $C \subseteq D$\\}

Como cualquier impar puede expresarse de la forma $2q + 1 : q \in \mathbb{N}$ entonces $D = \{n \in \mathbb{N} : n$ es impar$\}$.\\
\\
$ k \in \mathbb{N} \Rightarrow 4k + 1 = 2(2k) + 1$ (impares)\\

Notamos que $\mathbf{k = 0 \Rightarrow 4k + 1 = 1}$ y $\mathbf{k = 1 \Rightarrow 4k + 1 = 5}$
	\begin{center}
$\mathbf{\therefore 4k + 1 \neq 3}$\\
	\end{center}
Entonces $\forall x(x \in C \Rightarrow x$ es impar $\neq 3)$\\

Sabemos que $D$ contiene a todos los impares,\\ 
	
	\begin{center}
$\therefore C \subseteq D$
	\end{center}
	
\paragraph{$D \setminus C$\\}
$D \setminus C = \{x \in \mathbb{N} : x \in D \wedge x \notin C\}$\\
\subparagraph{$x \in D \Rightarrow x$ es impar.}
\subparagraph{$x \in C \Rightarrow x$ es impar $\neq 3$}
\begin{center}
$\therefore D \setminus C = \{k \in \mathbb{N} : 2k + 1 = 3\}$
\end{center}