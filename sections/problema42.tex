\section{}
Para cualesquiera conjuntos $A, B, C$ y $D$ tales que $A \neq  \varnothing$ y $C \neq \varnothing$, demuestre que $A \subseteq C$ y $C \subseteq D$ si y solo si $A x C \subseteq B x D$ ¿Qué sucede si alguno A o C es vacío?\newline

$(\rightarrow)$\newline
Dem Pd: Si $A \subseteq B$ y $C \subseteq D \rightarrow AxC \subseteq BxD$\newline
Sea $(a,c) \in AxC$, por hipótesis sabemos que $\forall x (x \in A \rightarrow x \in B) \wedge \forall x (x \in A \rightarrow x \in D) \rightarrow (a,c)$ también pertenece a $BxD$.\newline
$\therefore AxC \subseteq BxD$ \newline

$(\leftarrow)$\newline
Pd: $AxC \subseteq BxD \rightarrow A \subseteq B \wedge C \subseteq D$\newline
Caso I) Sean $A \neq \varnothing$ y $c \in C \rightarrow \exists$ la pareja $(a,c) \in AxC$, \newline
como $\forall (x,y)((x,y) \in AxC \rightarrow (x,y) \in BxD) \rightarrow (a,c) \in BxD$,\newline
luego $a \in B$ y $c \in D$\newline
$\therefore A \subseteq B \wedge C \subseteq D$\newline
Caso II) Este caso es análogo al caso I con la condición de que $a \in A$ y $c \neq \varnothing$\newline\\

¿Qué sucede si alguno A o C es vacío?\newline
En caso de que A o C sean igual al vacío, $AxD = \varnothing$, por lo que se cumpliría la condición de que $A \subseteq B$ y $C \subseteq D$ ($\forall A, \varnothing \subseteq A $ teorema visto en clase) pero no se cumplira $AxC \subseteq BxD$, ya que se define a  $BxD = \{ (b,d) | a \in A \wedge b \in B\}$ y si $AxC = \varnothing \rightarrow$ no existe una pareja $(b,d)$ que sea igual a  $\varnothing$ (solo su $B$ y $D \neq \varnothing$ ), y $\varnothing$ no puede ser una pareja $(b,d)$.\newline
Contraejemplo:\newline
Sea $A$ y $C = \varnothing$, $B = \{ 1\}$, $D = \{2\} \rightarrow AxC = \varnothing$ y $BD = \{\{1,2\} \}$, \newline
por lo que $AxC \subsetneqq BD\{\{1,2\}\} $  ya que el vacío no puede ser una pareja. 