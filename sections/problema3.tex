\section{}
Diga cuáles de los siguientes conjuntos es elemento el conjunto $\{ \varnothing \}$ y en cuáles está contenido, justificando sus respuestas:
\begin{enumerate}[(i)]
\item   $\{ \varnothing \} \in \varnothing$ \newline
			$\{ \varnothing \}$ no puede ser elemento de  $\varnothing$ pues por definición el conjunto vacío no tiene elementos. \newline\\
			$\{ \varnothing \} \subseteq \varnothing$\newline
			El elemento de $\{ \varnothing \}$ es el conjunto vacío y $ \varnothing  \notin \varnothing$. por lo tanto $\{\varnothing\}  \subsetneqq \varnothing$.
			
			\item $\{ \varnothing \} \in \{ \varnothing \}$ \newline
			$\{ \varnothing \}$ no es elemento del conjunto $\{ \varnothing \}$ ya que este último solo tiene como elemento a $\varnothing$ \newline \\
			$\{ \varnothing \} \subseteq \{ \varnothing \}$\newline
			Dado el Teorema 2.12 de Anti-simetría visto en clase, se cumple que $\{ \varnothing \} \subseteq \{ \varnothing \}$
			
			\item  $\{ \varnothing \} \in \{\{ \varnothing \}\}$ \newline
			$\{ \varnothing \}$ Si es un elemento del conjunto $\{\{ \varnothing \}\}$ pues el único elemento en el es justamente $\{ \varnothing \}$.\newline\\
			$\{ \varnothing \} \subseteq \{\{ \varnothing \}\}$\newline
			$\{ \varnothing \} \subsetneqq \{\{ \varnothing \}\}$ puesto que el único elemento del conjunto es $\{\varnothing\}$, si fuese solamente el conjunto vacío se cumpliría, pero no es el caso.
			
			\item  $\{ \varnothing \} \in \{ \varnothing , \{\{ \varnothing \}\}\}$\newline
			 $\{ \varnothing \}$ no es elemento de este conjunto, pues los únicos elementos que tiene son $\varnothing$ y  $\{\{ \varnothing \}\}$.\newline\\
			 $\{ \varnothing \} \subseteq \{ \varnothing , \{\{ \varnothing \}\}\}$\newline
			 Los elementos de este conjunto son $\varnothing$ y $\{\{ \varnothing \}\}$, y precisamente buscamos al conjunto vacío que es el único elemento del primer conjunto. Por lo que se cumple que  $\{ \varnothing \} \subseteq \{ \varnothing , \{\{ \varnothing \}\}\}$.
\end{enumerate}